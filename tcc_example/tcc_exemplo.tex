% ----------------------------------------------------------------------------------------------------- %
% Arquivo LaTeX de exemplo de TCC a ser apresentados à Ciência da Computação da UFT - Palmas
% 
% Versão 1.1:   Março 2016
%
% Criado por:   Tiago da Silva Almeida
% Revisado por: Tiago da Silva Almeida
%               Rafael Lima de Carvalho
%               Ary Henrique Morais de Oliveira
%
% http://uftex.sourceforge.net 
% ----------------------------------------------------------------------------------------------------- %

\documentclass[tcc2]{uftex}
	
\usepackage[num]{abntex2cite}
\renewcommand{\backrefpagesname}{}
\renewcommand{\backref}{}
\renewcommand*{\backrefalt}[4]{}

\makelosymbols
\makeloabbreviations

\begin{document}
  \title{Título do trabalho}
  \foreigntitle{Thesis Title}
  \author{Tiago da Silva}{Almeida}
  
  \advisor{Prof.}{José}{Mendonça}{Dr.}
  \advisor{Prof.}{Marcos}{da Oliveira}{Me.}

  \department{EC}
  \date{03}{2016}

  \keyword{Primeira palavra-chave}
  \keyword{Segunda palavra-chave}
  \keyword{Terceira palavra-chave}
  \keyword{Quarta palavra-chave}

  \foreignkeyword{First keyword}
  \foreignkeyword{Second keyword}
  \foreignkeyword{Third keyword}
  \foreignkeyword{Fourth keyword}

  \maketitle

  \frontmatter
  % ----------------------------------------------------------------------------------------------------- %
  %  Este trecho deve ser inserido somente no caso do TCC2 já na versão FINAL
  % ----------------------------------------------------------------------------------------------------- %
  %\includepdf{ficha_catalografica}
  %\includepdf{ata_de_aprovacao}
  % ----------------------------------------------------------------------------------------------------- %
  \dedication{A algu\'em cujo valor \'e digno desta dedicat\'oria.}

  \begin{acknowledgement}
  Gostaria de agradecer a todos.
  \end{acknowledgement}

  \begin{abstract}
  Apresenta-se, nesta tese, ...
  \end{abstract}

  \begin{foreignabstract}
  In this work, we present ...
  \end{foreignabstract}


\printlosymbols  
\printloabbreviations
\listoffigures            
\listoftables 
\tableofcontents 

\mainmatter
\onehalfspacing
% ----------------------------------------------------------------------------------------------------- %
% Capítulos do trabalho
% ----------------------------------------------------------------------------------------------------- %
\chapter{Introdução}
\label{cap:introducao}

\noindent Texto texto texto texto texto texto texto texto texto texto texto texto texto
texto texto texto texto texto texto texto texto texto texto texto texto texto
texto texto texto texto texto texto texto. \abbrev{IDE}{\textit{Integrated Development Environment}}\symbl{$\Pi$}{Produtório de algum valor}

Texto texto texto texto texto texto texto texto texto texto texto texto texto
texto texto texto texto texto texto texto texto texto texto texto texto texto
texto texto texto texto texto texto texto.

\begin{quote}
``Exemplo de citação direta com mais de três linhas''.
\end{quote}

\section{Exemplo de Código-Fonte em Java}
\label{sec:exemplo_codigo_fonte}

\noindent Texto texto texto texto texto texto texto texto texto texto texto texto texto
texto texto texto texto texto texto texto texto texto texto texto texto texto
texto texto texto texto texto texto texto texto texto texto texto texto texto
texto texto texto texto texto texto texto. \abbrev{ABNT}{Associação Brasileira de Normas Técnicas}\symbl{$\in$}{Pertence à}

\begin{lstlisting}[language=Java]
for(i = 0; i < 20; i++)
{
    // Comentario 
    System.out.println("Mensagem...");
}
\end{lstlisting}

\section{Exemplo de tabela}
\label{sec:exemplo_tabela}

\noindent Texto texto texto texto texto texto texto texto texto texto texto texto texto
texto texto texto texto texto texto texto texto texto texto texto texto texto
texto texto texto texto texto texto texto texto texto texto texto texto texto
texto texto texto texto texto texto texto \ref{tab:1}, \ref{tab:2}, \ref{tab:3} e \ref{tab:4}.

\begin{table}[!h]
\caption{Exemplo 1.}\label{tab:1}
\begin{tabular}{ l c r }
  1 & 2 & 3 \\
  4 & 5 & 6 \\
  7 & 8 & 9 \\
\end{tabular}
\end{table}

Texto texto texto texto texto texto texto texto texto texto texto texto texto
texto texto texto texto texto texto texto texto texto texto texto texto texto
texto texto texto texto texto texto texto texto texto texto texto texto texto
texto texto texto texto texto texto texto. \abbrev{PC}{\textit{Personal Computer}}\symbl{$\Sigma$}{Somatório de algum valor}

\begin{table}[!h]
\caption{Exemplo 2.}\label{tab:2}
\begin{tabular}{ |l|l| }
  \hline
  \multicolumn{2}{|c|}{Team sheet} \\
  \hline
  GK & Paul Robinson \\
  LB & Lucas Radebe \\
  DC & Michael Duberry \\
  DC & Dominic Matteo \\
  RB & Dider Domi \\
  MC & David Batty \\
  MC & Eirik Bakke \\
  MC & Jody Morris \\
  FW & Jamie McMaster \\
  ST & Alan Smith \\
  ST & Mark Viduka \\
  \hline
\end{tabular}
\end{table}

Texto texto texto texto texto texto texto texto texto texto texto texto texto
texto texto texto texto texto texto texto texto texto texto texto texto texto
texto texto texto texto texto texto texto texto texto texto texto texto texto
texto texto texto texto texto texto texto

\begin{table}[!h]
\caption{Exemplo 3.}\label{tab:3}
\begin{tabular}{l*{6}{c}r}
Team              & P & W & D & L & F  & A & Pts \\
\hline
Manchester United & 6 & 4 & 0 & 2 & 10 & 5 & 12  \\
Celtic            & 6 & 3 & 0 & 3 &  8 & 9 &  9  \\
Benfica           & 6 & 2 & 1 & 3 &  7 & 8 &  7  \\
FC Copenhagen     & 6 & 2 & 1 & 3 &  5 & 8 &  7  \\
\end{tabular}
\end{table}

Texto texto texto texto texto texto texto texto texto texto texto texto texto
texto texto texto texto texto texto texto texto texto texto texto texto texto
texto texto texto texto texto texto texto texto texto texto texto texto texto
texto texto texto texto texto texto texto

\begin{table}[!h]
\caption{Exemplo 4.}\label{tab:4}
\begin{center}
    \begin{tabular}{ | l | l | l | p{5cm} |}
    \hline
    Day & Min Temp & Max Temp & Summary \\ \hline
    Monday & 11C & 22C & A clear day with lots of sunshine.  
    However, the strong breeze will bring down the temperatures. \\ \hline
    Tuesday & 9C & 19C & Cloudy with rain, across many northern regions. Clear spells 
    across most of Scotland and Northern Ireland, 
    but rain reaching the far northwest. \\ \hline
    Wednesday & 10C & 21C & Rain will still linger for the morning. 
    Conditions will improve by early afternoon and continue 
    throughout the evening. \\
    \hline
    \end{tabular}
\end{center}
\end{table}

Texto texto texto texto texto texto texto texto texto texto texto texto texto
texto texto texto texto texto texto texto texto texto texto texto texto texto
texto texto texto texto texto texto texto texto texto texto texto texto texto
texto texto texto texto texto texto texto
%% ------------------------------------------------------------------------- %%
\chapter{Conclusões}
\label{cap:conclusoes}

\noindent Texto texto texto texto texto texto texto texto texto texto texto texto texto
texto texto texto texto texto texto texto texto texto texto texto texto texto
texto texto texto texto texto texto.


\backmatter 
\singlespacing   
% ----------------------------------------------------------------------------------------------------- %
% Bibliografia
% ----------------------------------------------------------------------------------------------------- %
\bibliography{tcc_exemplo}

% ----------------------------------------------------------------------------------------------------- %
% Anexos
% ----------------------------------------------------------------------------------------------------- %
\appendix
\onehalfspacing

\chapter{Sequências}
\label{ape:sequencias}

\noindent Texto texto texto texto texto texto texto texto texto texto texto texto texto
texto texto texto texto texto texto texto texto texto texto texto texto texto
texto texto texto texto texto texto.

\end{document}
